\documentclass[12pt]{article}
\usepackage{style}

\title{MATHEMATICAL ANALYSIS}
\author{Yaohui Wu}
\date{\today}

\begin{document}
\maketitle

\section{Real Numbers}
\subsection{The Set of Real Numbers}
We define
\begin{itemize}
    \item The set of natural numbers is \(\N=\{1,2,3,\dots\}\).
    \item The set of integers is \(\Z=\{0,1,-1,2,-2,\dots\}\).
    \item The set of rational numbers is
    \(\Q=\left\{\frac{m}{n}\mid m,n\in\Z,n\neq0\right\}\).
    \item The set of real numbers is \(\R\).
\end{itemize}
Note that \(\N\subset\Z\subset\Q\subset\R\).
\begin{definition}[Well Ordering Property of \(\N\)]
    Every nonempty subset of \(\N\) has a least element.
\end{definition}
\begin{definition}
    An \textit{ordered set} is a set \(S\) with a relation \(<\) such that
    \begin{enumerate}
        \item (\textit{Trichtomy}) If \(x,y\in S\), then one and only one of
        the statements \(x<y,x=y,y<x\) is true.
        \item (\textit{Transitivity}) If \(x,y,z\in S\), \(x<y\), and \(y<z\),
        then \(x<z\).
    \end{enumerate}
\end{definition}
\begin{definition}
    Suppose \(S\) is an ordered set and \(E\subset S\).
    If there exists a \(b\in S\) such that \(x\leq b\) for all \(x\in E\),
    then \(E\) is \textit{bounded above} and \(b\) is an \textit{upper bound}
    of \(E\).
    If there exists an upper bound \(a\) such that \(a\leq b\) for all upper
    bounds \(b\) of \(E\), then \(a\) is the \textit{least upper bound} or the
    \textit{supremum} of \(E\) denoted by \(\sup E\).
\end{definition}
Similar definition applies for a set that is \textit{bounded below} and
\textit{lower bound}.
The \textit{greatest lower bound} or the \textit{infimum} of \(E\) is denoted
by \(\inf E\).
A set is \textit{bounded} if it is bounded above and below.
\begin{definition}[Dedekind Completeness]
    An ordered set \(S\) has the \textit{least-upper-bound property} if every
    nonempty subset \(E\subset S\) that is bounded above has a least upper
    bound, that is, \(\sup E\) exists in \(S\).
\end{definition}
\begin{theorem}
    There exists a unique ordered field \(\R\) with the least-upper-bound
    property such that \(\Q\subset\R\).
\end{theorem}

\subsection{Archimedian Property}
\begin{theorem}
    If \(x,\varepsilon\in\R\) and \(x\leq\varepsilon\) for all
    \(\varepsilon>0\), then \(x\leq0\).
\end{theorem}
\begin{theorem}
    Suppose \(x,y\in\R\).
    \begin{enumerate}
        \item (Archimedian property) If \(x>0\), then there exists an
        \(n\in\N\) such that \(nx>y\).
        \item (\(\Q\) is dense in \(\R\)) If \(x<y\), then there exists an
        \(r\in\Q\) such that \(x<r<y\).
    \end{enumerate}
\end{theorem}
If \(A\subset\R\) and \(\sup A\in A\), then the supremum is the
\textit{maximum} of \(A\) denoted by \(\max A\).
Similarly, the infimum is the \textit{minimum} of \(A\) denoted by \(\min A\).
\begin{theorem}[Triangle Inequality]
    If \(x,y\in\R\), then \(|x+y|\leq|x|+|y|\).
\end{theorem}

\section{Sequences and Series}
\subsection{Sequences}
\begin{definition}
    A \textit{sequence} \(\{x_n\}\) is a function with domain \(\N\) and range
    \(\R\).
\end{definition}
\begin{definition}
    A sequence \(\{x_n\}\) is \textit{bounded} if there exists a \(B\in\R\)
    such that \(|x_n|\leq B\) for all \(n\in\N\).
\end{definition}
\begin{definition}
    A sequence is \textit{convergent} and has the unique \textit{limit} \(L\)
    and we write
    \[\lim_{n\to\infty}\{x_n\}=L\]
    if for every \(\varepsilon>0\), there exists an \(N\in\N\) such that
    \(|x_n-L|<\varepsilon\) for all \(n\geq N\).
\end{definition}
If a sequence is not convergent, then it is \textit{divergent}.
\begin{theorem}
    If a sequence is convergent, then it is bounded.
\end{theorem}
\begin{theorem}[Squeeze Theorem]
    If \(a_n\leq b_n\leq c_n\) for all \(n\geq N\), and
    \(\lim_{n\to\infty}a_n=\lim_{n\to\infty}c_n=L\), then
    \[\lim_{n\to\infty}b_n=L\]
\end{theorem}

\subsection{Monotonic Sequences}
\begin{definition}
    A sequence \(\{x_n\}\) is \textit{increasing} if \(x_n\leq x_{n+1}\) or
    \textit{decreasing} if \(x_n\geq x_{n+1}\) for all \(n\in\N\).
    A sequence is \textit{monotonic} if it is increasing or decreasing.
\end{definition}
\begin{theorem}[Monotone Convergence Theorem]
    A monotonic sequence is convergent if and only if it is bounded.
\end{theorem}
If \(\{x_n\}\) is increasing and bounded above, then
\[\lim_{n\to\infty}x_n=\sup x_n\]
If \(\{x_n\}\) is decreasing and bounded below, then
\[\lim_{n\to\infty}x_n=\inf x_n\]

\subsection{Subsequences}
\begin{definition}
    Let \(n_i\) be a sequence that satisfies \(n_i<n_{i+1}\) for all
    \(i\in\N\).
    The sequence \(\{x_{n_i}\}\) is a \textit{subsequence} of \(\{x_n\}\).
\end{definition}
\begin{theorem}
    If \(\{x_n\}\) is convergent, then every subsequence \(\{x_{n_i}\}\) is
    convergent, and
    \[\lim_{n\to\infty}x_n=\lim_{i\to\infty}x_{n_i}\]
\end{theorem}

\subsection{Bolzano-Weierstrass Theorem}
\begin{theorem}[Bolzano-Weierstrass Theorem]
    Every bounded sequence has a convergent subsequence.
\end{theorem}

\subsection{Cauchy Sequences}
\begin{definition}
    A sequence \(\{x_n\}\) is a \textit{Cauchy sequence} if for every
    \(\varepsilon>0\) there exists an \(N\in\N\) such that
    \(|x_m-x_n|<\varepsilon\) for all \(m,n\geq\N\).
\end{definition}
\begin{theorem}
    If a sequence is a Cauchy sequence, then it is bounded.
\end{theorem}
\begin{theorem}[Cauchy Completeness]
    A sequence is convergent if and only if it is a Cauchy sequence.
\end{theorem}

\end{document}
